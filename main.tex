%%
%% This is file `sample-sigplan.tex',
%% generated with the docstrip utility.
%%
%% The original source files were:
%%
%% samples.dtx  (with options: `sigplan')
%%
%% IMPORTANT NOTICE:
%%
%% For the copyright see the source file.
%%
%% Any modified versions of this file must be renamed
%% with new filenames distinct from sample-sigplan.tex.
%%
%% For distribution of the original source see the terms
%% for copying and modification in the file samples.dtx.
%%
%% This generated file may be distributed as long as the
%% original source files, as listed above, are part of the
%% same distribution. (The sources need not necessarily be
%% in the same archive or directory.)
%%
%%
%% Commands for TeXCount
%TC:macro \cite [option:text,text]
%TC:macro \citep [option:text,text]
%TC:macro \citet [option:text,text]
%TC:envir table 0 1
%TC:envir table* 0 1
%TC:envir tabular [ignore] word
%TC:envir displaymath 0 word
%TC:envir math 0 word
%TC:envir comment 0 0
%%
%%
%% The first command in your LaTeX source must be the \documentclass
%% command.
%%
%% For submission and review of your manuscript please change the
%% command to \documentclass[manuscript, screen, review]{acmart}.
%%
%% When submitting camera ready or to TAPS, please change the command
%% to \documentclass[sigconf]{acmart} or whichever template is required
%% for your publication.
%%
%%

\documentclass[sigplan,screen,review]{acmart}

\usepackage{paper_alcides}
\usepackage{cleveref}
\usepackage{pgfplots}
\usepackage{tikz}
\usetikzlibrary{decorations.pathreplacing}
\usetikzlibrary{positioning}
\usepgfplotslibrary{fillbetween}
\usepgflibrary{patterns}
\usetikzlibrary{patterns}

\usepackage{bussproofs}
\usepackage{listings}
%%
%% \BibTeX command to typeset BibTeX logo in the docs
\AtBeginDocument{%
  \providecommand\BibTeX{{%
    Bib\TeX}}}

%% Rights management information.  This information is sent to you
%% when you complete the rights form.  These commands have SAMPLE
%% values in them; it is your responsibility as an author to replace
%% the commands and values with those provided to you when you
%% complete the rights form.
\setcopyright{acmcopyright}
\copyrightyear{2018}
\acmYear{2018}
\acmDOI{XXXXXXX.XXXXXXX}

%% These commands are for a PROCEEDINGS abstract or paper.
\acmConference[Conference acronym 'XX]{Make sure to enter the correct
  conference title from your rights confirmation emai}{June 03--05,
  2018}{Woodstock, NY}
%%
%%  Uncomment \acmBooktitle if the title of the proceedings is different
%%  from ``Proceedings of ...''!
%%
%%\acmBooktitle{Woodstock '18: ACM Symposium on Neural Gaze Detection,
%%  June 03--05, 2018, Woodstock, NY}
\acmPrice{15.00}
\acmISBN{978-1-4503-XXXX-X/18/06}


%%
%% Submission ID.
%% Use this when submitting an article to a sponsored event. You'll
%% receive a unique submission ID from the organizers
%% of the event, and this ID should be used as the parameter to this command.
%%\acmSubmissionID{123-A56-BU3}

%%
%% For managing citations, it is recommended to use bibliography
%% files in BibTeX format.
%%
%% You can then either use BibTeX with the ACM-Reference-Format style,
%% or BibLaTeX with the acmnumeric or acmauthoryear sytles, that include
%% support for advanced citation of software artefact from the
%% biblatex-software package, also separately available on CTAN.
%%
%% Look at the sample-*-biblatex.tex files for templates showcasing
%% the biblatex styles.
%%

%%
%% The majority of ACM publications use numbered citations and
%% references.  The command \citestyle{authoryear} switches to the
%% "author year" style.
%%
%% If you are preparing content for an event
%% sponsored by ACM SIGGRAPH, you must use the "author year" style of
%% citations and references.
%% Uncommenting
%% the next command will enable that style.
%%\citestyle{acmauthoryear}


%%
%% end of the preamble, start of the body of the document source.
\begin{document}

%%
%% The "title" command has an optional parameter,
%% allowing the author to define a "short title" to be used in page headers.
\title{Implementing Separation Logic using an SMT-backed Frame Rule}

%%
%% The "author" command and its associated commands are used to define
%% the authors and their affiliations.
%% Of note is the shared affiliation of the first two authors, and the
%% "authornote" and "authornotemark" commands
%% used to denote shared contribution to the research.
\author{Kirill Golubev}
%\email{trovato@corporation.com}
% TODO: \orcid{1234-5678-9012}
\author{Alcides Fonseca}
\email{alcides@ciencias.ulisboa.pt}
\orcid{0000-0002-0879-4015}
\affiliation{%
  \institution{LASIGE, Faculdade de Ciências da Universidade de Lisboa}
  \city{Lisboa}
  \country{Portugal}
}

%%
%% The abstract is a short summary of the work to be presented in the
%% article.
\begin{abstract}

Symbolic execution is a technique frequently used to reason about code. In symbolic execution, the analyser keeps track of a logical representation of state, and correctness verifications are SMT queries. Separation logic is frequently used to express and verify properties of programs with pointers or references. However, most SMT solvers (like the popular z3) do not support Separation Logic natively. CVC5 has introduced partial support for separation logic, which has not yet been integrated into a more high-level tools.

This work aims to address this gap, by providing a proof of concept for implementing the Frame Rule using SMT queries in the Symbolic Heap fragment of separation logic, supported by CVC5. We conclude that this encoding can simplify the machinery dealing with separation logic, such as that present in Viper, Smallfoot, and others.

\todo {Viper is a debatable example, as it does not use separation logic internally. Instead it relies on more powerful mechanism of Implicit Dynamic Frames.}

\end{abstract}

%%
%% The code below is generated by the tool at http://dl.acm.org/ccs.cfm.
%% Please copy and paste the code instead of the example below.
%%
\begin{CCSXML}
<ccs2012>
 <concept>
  <concept_id>10010520.10010553.10010562</concept_id>
  <concept_desc>Computer systems organization~Embedded systems</concept_desc>
  <concept_significance>500</concept_significance>
 </concept>
 <concept>
  <concept_id>10010520.10010575.10010755</concept_id>
  <concept_desc>Computer systems organization~Redundancy</concept_desc>
  <concept_significance>300</concept_significance>
 </concept>
 <concept>
  <concept_id>10010520.10010553.10010554</concept_id>
  <concept_desc>Computer systems organization~Robotics</concept_desc>
  <concept_significance>100</concept_significance>
 </concept>
 <concept>
  <concept_id>10003033.10003083.10003095</concept_id>
  <concept_desc>Networks~Network reliability</concept_desc>
  <concept_significance>100</concept_significance>
 </concept>
</ccs2012>
\end{CCSXML}

\ccsdesc[500]{Computer systems organization~Embedded systems}
\ccsdesc[300]{Computer systems organization~Redundancy}
\ccsdesc{Computer systems organization~Robotics}
\ccsdesc[100]{Networks~Network reliability}


\received{20 February 2007}
\received[revised]{12 March 2009}
\received[accepted]{5 June 2009}

%%
%% This command processes the author and affiliation and title
%% information and builds the first part of the formatted document.
\maketitle


\newcommand{\EM}[1]{\ensuremath{#1}}
\newcommand{\ssymbol}[1]{\EM{#1}}
\newcommand{\bnfdef}{\EM{\vcentcolon\vcentcolon=}}
\newcommand{\emphbf}[1]{\textbf{\emph{#1}}}
\newcommand{\spmid}{\EM{\ \mid \ }}
\newcommand{\vsample}[1]{\EM{\mathit{sample}(#1)}}
\newcommand{\vconst}{\EM{\mathsf{c}}}
\newcommand{\anyval}{\ssymbol{v}}
\newcommand{\anydist}{\ssymbol{d}}
\newcommand{\dnormal}[2]{\EM{\mathit{Normal}(#1,#2)}}
\newcommand{\duniform}[2]{\EM{\mathit{Uniform}(#1,#2)}}

\section{Introduction} 

\todo{Introduce the topic of Program Verification.}

\todo{Introduce the topic of Separation Logic}

\todo{Symbolic heap}

\todo{GRASS}

\todo{Mention tools that use separation logic, and 
describe how they are encoded.}

\todo{SMT solver as an Oracle}

\todo{Describe the partial support of SL in CVC5, and identify that z3 has no such support.}


The goal of the present work is to provide an opportunity to shift some heavy lifting related to separation logic from a symbolic execution engine to an SMT solver. 

This is done by means of providing an algorithm to encode the frame rule through SMT solver queries. 


\section{Frame Rule}



The algorithm uses the notion of SMT query that is denoted as follows.

\[
\texttt{isUNSAT($\neg$query) = true,}
\]

iff an SMT solver gives the UNSAT result on the negation of the query. This means that for all free variables negation of the query does not hold, which is equivalent to the situation when the query holds for all free variables. 

\begin{prooftree}
    \AxiomC{\texttt{\{pre\} code \{post\}}}
    \UnaryInfC{\texttt{\{pre * frame\} code \{post * frame\}}}
\end{prooftree}

The algorithm itself is split into two phases. 

\begin{itemize}
    \item Check if the current context satisfies the precondition
    \item Apply postcondition to the larger context
\end{itemize}

Pseudocode for the first phase is quite simple. It exploits the idea, that it is possible to encode a heap containing any given one by adding \texttt{* true} to it. 

The outline is that during the first step, it checks if the boolean context, defining pointer equivalence, and the heap imply precondition.

\begin{lstlisting}[mathescape]
BCtx | H $\Vdash$ pre 
 $iff$ isUNSAT($\neg$(BCtx $\land$ H $\implies$ pre * true))
\end{lstlisting}

The second phase exploits the same idea. The frame is inferred by checking each pointer for belonging in a frame, by precondition invalidation.

\begin{lstlisting}[mathescape]
H = h$_0$ * $\dots$ * h$_{n - 1}$ //larger context
frame = H.map($\lambda$ h. pre * h * true)
            .filter($\lambda$ c. BCtx | H $\Vdash$ c)
            .fold(emp, *)

\end{lstlisting}

The unfortunate consequence of this approach is a performance hit. Usually, the systems that are using SMT solvers need only \texttt{O(1)} SMT queries to make the symbolic execution step, but this algorithm does it in \texttt{O(size(H))} queries. 

If examined more closely, this algorithm is not exactly doing frame rule application, but rather heap reconstruction. It will discard every heaplet that invalidates precondition and the remainder will be the result. This makes it possible to use it for other purposes with slight variations. For example, for merging heaps after branching. 

\section{Conclusions}

One big advantage of this approach is that the SMT solver algorithm for separation logic is decidable. In contrast to Viper which is a notable alternative to writing a symbolic execution engine from scratch. 

The simplicity and decidability come with the cost of features that are possible to support. Viper is a much more mature and rich backend for the language, while the presented approach is capped by the capabilities of separation logic support in SMT solver. Said capabilities are defined by GRASS fragment of separation logic which looks like "propositional" separation logic. The main features that are kept unreachable by this limitation are recursive predicates and fractional permissions. 

The primary target of this algorithm was Liquid Java, but it is general enough to be used in other projects relying on SMT solvers to verify symbolic execution steps. The primary benefit of this algorithm is simplicity and delegation of responsibility for separation logic handling to the SMT solver instead of a symbolic execution engine which is usually implemented separately for each tool. 

We implemented and tested this algorithm for Liquid Java, preliminary results are encouraging feature- and performance-wise. The prototype supports function calls, conditional branching, and assignments.

The performance degradation for synthetic benchmarks is around 30\% relative to the pure boolean version of Liquid Java.

\bibliographystyle{acm}
\bibliography{main}

\end{document}
\endinput
%%
%% End of file `sample-sigplan.tex'.
