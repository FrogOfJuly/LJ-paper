%%
%% This is file `sample-sigplan.tex',
%% generated with the docstrip utility.
%%
%% The original source files were:
%%
%% samples.dtx  (with options: `sigplan')
%%
%% IMPORTANT NOTICE:
%%
%% For the copyright see the source file.
%%
%% Any modified versions of this file must be renamed
%% with new filenames distinct from sample-sigplan.tex.
%%
%% For distribution of the original source see the terms
%% for copying and modification in the file samples.dtx.
%%
%% This generated file may be distributed as long as the
%% original source files, as listed above, are part of the
%% same distribution. (The sources need not necessarily be
%% in the same archive or directory.)
%%
%%
%% Commands for TeXCount
%TC:macro \cite [option:text,text]
%TC:macro \citep [option:text,text]
%TC:macro \citet [option:text,text]
%TC:envir table 0 1
%TC:envir table* 0 1
%TC:envir tabular [ignore] word
%TC:envir displaymath 0 word
%TC:envir math 0 word
%TC:envir comment 0 0
%%
%%
%% The first command in your LaTeX source must be the \documentclass
%% command.
%%
%% For submission and review of your manuscript please change the
%% command to \documentclass[manuscript, screen, review]{acmart}.
%%
%% When submitting camera ready or to TAPS, please change the command
%% to \documentclass[sigconf]{acmart} or whichever template is required
%% for your publication.
%%
%%

\documentclass[sigplan,screen,review]{acmart}


\usepackage{paper_alcides}
\usepackage{cleveref}
\usepackage{pgfplots}
\usepackage{tikz}
\usetikzlibrary{decorations.pathreplacing}
\usetikzlibrary{positioning}
\usepgfplotslibrary{fillbetween}
\usepgflibrary{patterns}
\usetikzlibrary{patterns}
\usepackage{bussproofs}
\usepackage{listings}
\usepackage{bbding}
%%
%% \BibTeX command to typeset BibTeX logo in the docs
\AtBeginDocument{%
  \providecommand\BibTeX{{%
    Bib\TeX}}}

%% Rights management information.  This information is sent to you
%% when you complete the rights form.  These commands have SAMPLE
%% values in them; it is your responsibility as an author to replace
%% the commands and values with those provided to you when you
%% complete the rights form.
\setcopyright{acmcopyright}
\copyrightyear{2018}
\acmYear{2018}
\acmDOI{XXXXXXX.XXXXXXX}

%% These commands are for a PROCEEDINGS abstract or paper.
\acmConference[Conference acronym 'XX]{Make sure to enter the correct
  conference title from your rights confirmation emai}{June 03--05,
  2018}{Woodstock, NY}
%%
%%  Uncomment \acmBooktitle if the title of the proceedings is different
%%  from ``Proceedings of ...''!
%%
%%\acmBooktitle{Woodstock '18: ACM Symposium on Neural Gaze Detection,
%%  June 03--05, 2018, Woodstock, NY}
\acmPrice{15.00}
\acmISBN{978-1-4503-XXXX-X/18/06}


%%
%% Submission ID.
%% Use this when submitting an article to a sponsored event. You'll
%% receive a unique submission ID from the organizers
%% of the event, and this ID should be used as the parameter to this command.
%%\acmSubmissionID{123-A56-BU3}

%%
%% For managing citations, it is recommended to use bibliography
%% files in BibTeX format.
%%
%% You can then either use BibTeX with the ACM-Reference-Format style,
%% or BibLaTeX with the acmnumeric or acmauthoryear sytles, that include
%% support for advanced citation of software artefact from the
%% biblatex-software package, also separately available on CTAN.
%%
%% Look at the sample-*-biblatex.tex files for templates showcasing
%% the biblatex styles.
%%

%%
%% The majority of ACM publications use numbered citations and
%% references.  The command \citestyle{authoryear} switches to the
%% "author year" style.
%%
%% If you are preparing content for an event
%% sponsored by ACM SIGGRAPH, you must use the "author year" style of
%% citations and references.
%% Uncommenting
%% the next command will enable that style.
%%\citestyle{acmauthoryear}


%%
%% end of the preamble, start of the body of the document source.
\begin{document}

%%
%% The "title" command has an optional parameter,
%% allowing the author to define a "short title" to be used in page headers.
\title[Implementing Separation Logic using an SMT-backed Frame Rule]{Implementing Separation Logic \\ using an SMT-backed Frame Rule}

%%
%% The "author" command and its associated commands are used to define
%% the authors and their affiliations.
%% Of note is the shared affiliation of the first two authors, and the
%% "authornote" and "authornotemark" commands
%% used to denote shared contribution to the research.
\author{Kirill Golubev}
\email{gkigorevich@ciencias.ulisboa.pt}
\orcid{0009-0002-2709-5241}

\author{Alcides Fonseca}
\email{alcides@ciencias.ulisboa.pt}
\orcid{0000-0002-0879-4015}
\affiliation{%
  \institution{LASIGE, Faculdade de Ciências da Universidade de Lisboa}
  \city{Lisboa}
  \country{Portugal}
}

%%
%% The abstract is a short summary of the work to be presented in the
%% article.
\begin{abstract}

Symbolic execution is a technique frequently used to reason about code. In symbolic execution, an analyzer keeps track of the program using a representation of the logical state, and validates that state transitions are valid. This verification is often discharged to SMT solvers as queries to the logical state.

Separation Logic is frequently used to express and verify the properties of programs with pointers or references. However, most SMT solvers (like the popular z3~\cite{DBLP:conf/tacas/MouraB08}) do not support Separation Logic natively. Recently, the CVC5 SMT Solver has introduced partial support for separation logic, which has not yet been integrated into more high-level tools.

This work aims to address this gap, by providing a proof of concept for implementing the Frame Rule using SMT queries in the Symbolic Heap fragment of Separation Logic, as supported by CVC5. We conclude that this encoding can simplify the machinery dealing with separation logic, such as that present in Smallfoot or Verifast.

\end{abstract}

%%
%% The code below is generated by the tool at http://dl.acm.org/ccs.cfm.
%% Please copy and paste the code instead of the example below.
%%
\begin{CCSXML}
<ccs2012>
 <concept>
  <concept_id>10010520.10010553.10010562</concept_id>
  <concept_desc>Computer systems organization~Embedded systems</concept_desc>
  <concept_significance>500</concept_significance>
 </concept>
 <concept>
  <concept_id>10010520.10010575.10010755</concept_id>
  <concept_desc>Computer systems organization~Redundancy</concept_desc>
  <concept_significance>300</concept_significance>
 </concept>
 <concept>
  <concept_id>10010520.10010553.10010554</concept_id>
  <concept_desc>Computer systems organization~Robotics</concept_desc>
  <concept_significance>100</concept_significance>
 </concept>
 <concept>
  <concept_id>10003033.10003083.10003095</concept_id>
  <concept_desc>Networks~Network reliability</concept_desc>
  <concept_significance>100</concept_significance>
 </concept>
</ccs2012>
\end{CCSXML}

\ccsdesc[500]{Computer systems organization~Embedded systems}
\ccsdesc[300]{Computer systems organization~Redundancy}
\ccsdesc{Computer systems organization~Robotics}
\ccsdesc[100]{Networks~Network reliability}


\received{20 February 2007}
\received[revised]{12 March 2009}
\received[accepted]{5 June 2009}

%%
%% This command processes the author and affiliation and title
%% information and builds the first part of the formatted document.
\maketitle


\newcommand{\EM}[1]{\ensuremath{#1}}
\newcommand{\ssymbol}[1]{\EM{#1}}
\newcommand{\bnfdef}{\EM{\vcentcolon\vcentcolon=}}
\newcommand{\emphbf}[1]{\textbf{\emph{#1}}}
\newcommand{\spmid}{\EM{\ \mid \ }}
\newcommand{\vsample}[1]{\EM{\mathit{sample}(#1)}}
\newcommand{\vconst}{\EM{\mathsf{c}}}
\newcommand{\anyval}{\ssymbol{v}}
\newcommand{\anydist}{\ssymbol{d}}
\newcommand{\dnormal}[2]{\EM{\mathit{Normal}(#1,#2)}}
\newcommand{\duniform}[2]{\EM{\mathit{Uniform}(#1,#2)}}

\lstset{mathescape}

\section{Introduction} 

Formal verification is one of the few known ways to ensure that a computer program has the least amount of errors. The idea is to prove that a given program satisfies a specification provided in advance. The language of program specification is usually some logic that fits to describe program behavior. There are many ways to achieve this with varying resource requirements and automation degrees.

One technique employed to verify imperative programs is symbolic execution~\cite{berdine2005symbolic}. Usually, the engine for it is implemented separately for each tool~\cite{DBLP:conf/fmco/BerdineCO05,DBLP:conf/oopsla/DistefanoP08}. SMT solvers are used as oracles to verify that each step of the symbolic execution engine is correct. 

Most programs in imperative languages are written in terms of heap manipulation. Separation Logic~\cite{DBLP:journals/cacm/OHearn19} is an extension of Hoare Logic~\cite{DBLP:journals/cacm/Hoare69}, frequently used to reason about these types of programs. It introduces some new operations and constants to it alongside a new inference rule called \emph{Frame Rule}. In general, Separation Logic is proved to be undecidable~\cite{DBLP:conf/lics/BrotherstonK10}, but there are decidable subsets.

In this work, we focus on the \emph{Symbolic Heap} fragment~\cite{DBLP:conf/fmco/BerdineCO05}, as it enables a significant degree of automation by being supported in the CVC5 SMT Solver~\cite{DBLP:conf/tacas/BarbosaBBKLMMMN22}. A simplified Symbolic Heap requires the logical context to be split into a pure boolean part ($\mbox{BCtx}$) and a pure spatial part ($H = emp * h_0 * \dots * h_n$), made of disjoint heaplets ($h_i = p \mapsto v$)

\[
\text{BCtx} \wedge \text{emp} * h_0 * \dots* h_n
\]

This simplified version significantly restricts the expressive power of Separation Logic but still permits encoding of some interesting program properties. Thus, we chose it as a starting point, but CVC5 actually supports a larger fragment called GRASS~\cite{DBLP:conf/cav/PiskacWZ13}. In contrast, Z3 does not have any support for separation logic.~\footnote{But it was once prototyped: \url{https://github.com/Z3Prover/z3/issues/811}.} 

\begin{figure}
\begin{prooftree}
    \AxiomC{\texttt{\{pre\} code \{post\}}}
    \RightLabel{Frame rule}
    \UnaryInfC{\texttt{\{pre * frame\} code \{post * frame\}}}
\end{prooftree}
\caption{The general definition of the Frame Rule.}
\label{fig:framerule}
\end{figure}

The goal of this work is to show evidence of the opportunity to shift some heavy lifting related to separation logic from a symbolic execution engine to an SMT solver. For this, we present an algorithm to encode the Frame Rule of Separation Logic (\Cref{fig:framerule}) through SMT Solver queries.

\section{Frame Rule}

\paragraph{Preliminaries} To prove that a formula with universally free variables holds, we query the SMT solver whether the negation of the query, using existentially quantified free variables, is unsatisfiable (\texttt{unsat}). We denote this verification as $\operatorname{isUNSAT}(\neg\mbox{query})$.


\paragraph{Algorithm} The algorithm to check the Frame Rule is split into two phases: 

\begin{itemize}
    \item Check if the current context satisfies the precondition
    \item Apply postcondition to the larger context
\end{itemize}


The first step focus on verifying whether the precondition is guaranteed by the context. The context is composed of the pure boolean context($\mbox{BCtx}$), including pointer equivalente (e.g., $a=b$) and the heap (e.g., $a \mapsto x * b \mapsto y$). The requirement is any heap, containing precondition: ($\text{pre} * \text{true}$).

\begin{align*}
\mbox{BCtx} \wedge H \Vdash \mbox{pre} & \, \iff & \\ & \operatorname{isUNSAT}(\neg(\mbox{BCtx} \wedge H \implies \mbox{pre} \mathbin{*} \mbox{true} ))
\end{align*}

%\begin{lstlisting}[mathescape]
%BCtx | H $\Vdash$ pre 
% $iff$ isUNSAT($\neg$(BCtx $\land$ H $\implies$ pre * true))
%\end{lstlisting}


The second step isolates the unchanged part of the heap, called $\mbox{frame}$, to be kept in the outgoing context, by discarding all heaplets that invalidates the precondition. In particular, for the incoming context $S_{in} = \mbox{BCtx} \wedge H_{in} = \mbox{BCtx} \wedge (\text{pre} * \text{frame})$, we will generate the outgoing context $S_{out} = \text{BCtx} \wedge H_{out} = \text{BCtx} \wedge (\text{post} * \text{frame})$. 

\begin{align*}
&H_{in} = h_1 * \dots * h_n\\
&\text{frame} = \* \{ h_i  \,|\, i \in 1\dots n,\, \mbox{BCtx} \wedge H_{in} \Vdash \mbox{pre} \mathbin{*} h_i  \mathbin{*} \text{true} \}\\
&H_{out} = post * \prod_{h_j \in frame} h_j
\end{align*}

Where $\prod$ is used with respect to separating conjunction. From this heap reconstruction, we can build the complete outgoing context $S_{out}$. The heap reconstruction process can be used in other contexts, such as merging heaps after conditional branching.

When compared to other SMT-based Separation Logic approach, such as Implicit Dynamic Frames, our approach takes \texttt{O(size(H))} queries from decidable logics, instead of the \texttt{O(1)} queries in undecidable logics. This approach presents another alternative in the design space between guarantee of results versus performance.

\todo{I think it is better to have concrete examples. I have no evidence that undecidablitiy is connected with O(1) queries per frame rule invlocation}

\paragraph{Evaluation} We validated the feasibility of this approach by implementing it in the \href{https://catarinagamboa.github.io/liquidjava.html}{Liquid Java} compiler, supporting function calls, conditional branching, and assignments. The performance degradation for synthetic benchmarks is around 30\% relative to the pure boolean version of Liquid Java.


\section{Conclusions}

The big advantage of this approach is that the SMT solver algorithm for separation logic is decidable. This is different than other approaches, such as Viper~\cite{DBLP:conf/vmcai/0001SS16}, which have their own internal infrastructure for implementing the frame rule.

\todo{Viper does not implement frame rule at all}

The simplicity and decidability come with the cost of features that are possible to support. Viper is a much more mature and rich backend for the language, while the presented approach is capped by the capabilities of separation logic support in SMT solver. Said capabilities are defined by GRASS fragment of separation logic which looks like ``propositional'' separation logic. The main features that are kept unreachable by this limitation are recursive predicates and fractional permissions~\cite{DBLP:conf/sas/Boyland03}. 

While our validation was in the specific context of Liquid Java~\cite{gamboa2021user}, but it is general enough to be used in other projects relying on SMT solvers to verify symbolic execution steps. The primary benefit of this algorithm is simplicity and delegation of responsibility for separation logic handling to the SMT solver instead of a symbolic execution engine which is usually implemented separately for each tool. 


\bibliographystyle{acm}
\bibliography{main}

\end{document}
\endinput
%%
%% End of file `sample-sigplan.tex'.